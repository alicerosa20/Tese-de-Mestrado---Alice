% #############################################################################
% RESUMO em Português
% !TEX root = ../main.tex
% #############################################################################
% use \noindent in firts paragraph
% reset acronyms
\acresetall
\noindent Nos últimos anos, métodos de aprendizagem profunda têm sido estudados para a segmentação de imagens microscópicas com o objectivo de qualificar e caracterizar células, núcleos e outras estruturas biológicas de forma eficiente e exata, uma vez que, esta informação é essencial, por exemplo, para o diagnóstico de doenças.

Neste trabalho, aborda-se a segmentação de núcleos e Golgi das células em imagens microscópicas tridimensionais (3D) fluorescentes como um problema de tradução de imagem para imagem desemparelhadas. Para a realização desta tarefa, nós propomos a utilização do modelo CycleGAN uma vez que pode ser treinado de forma \textit{end-to-end} não-supervisionada. Com este modelo é possível reduzir significativamente o tempo de treino e de teste uma vez que não requer a utilização de máscaras de segmentação manualmente anotadas para o seu treino, em vez disso é treinado com máscaras sintetizadas.

Adicionalmente, a deteção de pares de núcleo-Golgi é também importante para a análise de imagens microscópicas para diagnósticos, logo, o modelo proposto é extendido de modo a classificar uma terceira classe, os pares núcleo-Golgi.

Os resultados experimentais obtidos com o modelo CycleGAN proposto são comparados com os obtidos com outros dois modelos de segmentação supervisionados, U-Net 3D e Vox2Vox. Com o modelo CycleGAN obteve-se os seguintes resultados: coeficiente Dice de 76,64\% para a classe do núcleo, 64,27\% para a classe do Golgi e 69,99\% para a classe de pares núcleo-Golgi, com uma diferença de apenas 1,11\%, 5,11\% e 6,02\%, respectivamente, dos melhores resultados obtidos com o modelo supervisionado U-Net 3D. Adicionalmente, o tempo necessário para treinar e testar o modelo U-Net é, aproximadamente, 2,63 vezes mais longo do que o tempo necessário para treinar e testar o modelo CycleGAN.



